\documentclass[]{scrreprt}

\begin{document}
\chapter{Fitting basics}
The Time Tagger acquires clock time-tags $y_i$ which are simply enumerated by an index $x_i$. The time-tags shall be fitted by a function
\begin{equation}
    f(x) = ax + b
\end{equation}
by least squares. For a set of $N$ time-tags, we need to minimize
\begin{equation}
    g(a, b) = \sum_{i=0}^{N-1}\left(y_i - f(x_i)\right)^2
            = \sum_{i=0}^{N-1}\left(y_i - a x_i - b\right)^2
\end{equation}
so the coefficients $a$ and $b$ need to fulfill the conditions
\begin{eqnarray}
    \partial_a g &=& -2 \sum_{i=0}^{N-1} \left(y_i - a x_i - b\right) x_i = 0 \label{eq:condition_a}\\
    \partial_b g &=& -2 \sum_{i=0}^{N-1} \left(y_i - a x_i - b\right) = 0.\label{eq:condition_b}
\end{eqnarray}
Equation \ref{eq:condition_b} directly results in
\begin{equation}
    b = \frac{1}{N} \sum_{i=0}^{N-1} \left(y_i - a x_i\right) = \frac{S_y - a S_x}{N} \label{eq:b_depending_on_a}
\end{equation}
where we used the symbols
\begin{eqnarray}
    S_x &=& \sum_{i=0}^{N-1} x_i \nonumber \\
    S_y &=& \sum_{i=0}^{N-1} y_i \nonumber \\
    S_{xx} &=& \sum_{i=0}^{N-1} x_i^2 \nonumber \\
    S_{xy} &=& \sum_{i=0}^{N-1} x_i y_i. \nonumber
\end{eqnarray}
If we put \ref{eq:b_depending_on_a} into \ref{eq:condition_a}, we obtain
\begin{equation}
    S_{xy} - a S_{xx} - b S_x = S_{xy} - a S_{xx} - \frac{S_y - a S_x}{N} S_x = 0
\end{equation}
which results together with \ref{eq:b_depending_on_a} in
\begin{eqnarray}
    a &=& \frac{S_x S_y - N S_{xy}}{S_x^2 - N S_{xx}} \\
    b &=& \frac{S_{xy} S_x - S_y S_{xx}}{S_x^2 - N S_{xx}}. \label{eq:b_basic}
\end{eqnarray}

\chapter{Moving fit}
The goal is now to use the offset $b$ to correct the position of the current clock tag. For this purpose, the fit is performed over the last $N$ time-tags. The current time-tag will always get the index $i=0$ and will by definition be set to $x_0 \equiv 0$ and $y_0 \equiv 0$. The preceding tag is set to $x_1 = -1$ and so on, so the fit will always be performed over the $x$-range $[0, -1, \dots, -(N-1)]$ which results in
\begin{eqnarray}
    S_x &=& \sum_{i=0}^{N-1} x_i = \sum_{i=0}^{N-1} -i = - \frac{N (N-1)}{2} \\
    S_{xx} &=& \sum_{i=0}^{N-1} x_i^2 = \sum_{i=0}^{N-1} (-i)^2 = \frac{(N-1) N (2N - 1)}{6}.
\end{eqnarray}
This simplifies expression \ref{eq:b_basic}
\begin{equation}
    b = \frac{(2N - 1) S_y - 3 S_{xy}}{\sum_{i=1}^N i}.
\end{equation}
Because it is not a good idea to perform the calculation of $S_y$ and $S_{xy}$ for every clock tag from the set of $N$ tags, we will calculate it incrementally. Here we need to distinguish two cases: At the beginning of the fit, when less than $N$ tags already arrived, the set will grow with every new tag. If there are $N$ tags in the set, the oldest one needs to be dropped when a new tag arrives.
\par To keep numbers small, only relative values are used. Additionally, they are reduced by the expected period $T$.
\section{Growing set of tags}
We start with a set of size $N$. When the tag $i=n+1$ arrives, it will grow to a size of $N+1$ and the value $S_y$ will change ($\Delta_+$ indicates the growing case) by:
\begin{eqnarray}
    \Delta_+ S_y^{(n+1)} &=& \left(\sum_{i=0}^N y_{n + 1 - i} - y_{n+1} + i T\right) - \left(\sum_{i=0}^{N-1} y_{n - i} - y_n + i T\right) \label{eq:DeltaPlus_S_y_explicit} \\
    &=& - N \left(y_{n+1} - y_n - T\right) \nonumber \\
    &=& - N d_{n+1}
\end{eqnarray}
with the reduced step size $d_{n+1} = y_{n+1} - y_n - T$. Similarly we can calculate the increment of $S_{xy}$:
\begin{eqnarray}
    \Delta_+ S_{xy}^{(n+1)} &=& \left(\sum_{i=0}^N -i \left(y_{n + 1 - i} - y_{n+1} + i T\right)\right) - \left(\sum_{i=0}^{N-1} -i \left(y_{n - i} - y_n + i T\right)\right) \label{eq:DeltaPlus_S_xy_explicit} \\
    &=& \left(\sum_{i=0}^{N-1}-y_{n-i}\right) + y_{n+1} \left(\sum_{i=0}^N i\right) - y_n \left(\sum_{i=0}^{N-1} i\right) - N^2 T \nonumber \\
    &=& -\left(\sum_{i=0}^{N-1} y_{n-i} - y_n + iT\right) + y_{n+1} \left(\sum_{i=0}^N i\right) - y_n \left(\sum_{i=0}^N i\right) - \left(N^2 - \sum_{i=0}^{N-1} i\right) T \nonumber \\
    &=& -S_y^{(n)} + \left(y_{n+1} - y_n\right) \left(\sum_{i=0}^N i\right) - \left(N^2 - \frac{N(N-1)}{2}\right) T \nonumber \\
    &=& -S_y^{(n)} + \left(y_{n+1} - y_n\right) \left(\sum_{i=0}^N i\right) - \left(\frac{N(N+1)}{2}\right) T \nonumber \\
    &=& -S_y^{(n)} + \left(y_{n+1} - y_n - T\right) \left(\sum_{i=0}^N i\right) \nonumber \\
    &=& -S_y^{(n)} + d_{n+1} \left(\sum_{i=0}^N i\right)
\end{eqnarray}

\section{Constant set size}
If the set is full, the calculation is very similar, but in both cases, eqs. \ref{eq:DeltaPlus_S_y_explicit} and \ref{eq:DeltaPlus_S_xy_explicit}, the sum in the first summand does not run up to $N$ but only up to $N-1$. The summand for $i=N$ is simply subtracted:
\begin{eqnarray}
    \Delta S_y^{(n+1)} &=& \left(\sum_{i=0}^{N-1} y_{n + 1 - i} - y_{n+1} + i T\right) - \left(\sum_{i=0}^{N-1} y_{n - i} - y_n + i T\right) \nonumber \\
    &=& \Delta_+ S_y^{(n+1)} - \left(y_{n+1-N} - y_{n+1} + NT\right) \nonumber \\
    &=& - N d_{n+1} + D_{n+1},
\end{eqnarray}
\begin{eqnarray}
    \Delta S_{xy}^{(n+1)} &=& \left(\sum_{i=0}^{N-1} -i \left(y_{n+1-i} - y_{n+1} + i T\right)\right) - \left(\sum_{i=0}^{N-1} -i \left(y_{n - i} - y_n + i T\right)\right) \nonumber \\
    &=& \Delta_+ S_{xy}^{(n+1)} - \left(-N\left(y_{n+1-N} - y_{n+1} + NT\right)\right) \nonumber \\
    &=& \Delta_+ S_{xy}^{(n+1)} - N D_{n+1}.
\end{eqnarray}
Here, $D_{n+1}$ is the time between the newly added tag and the one that is dropped.

\chapter{Python naming}
The following names are used as attribute in the Python class {\ttfamily LinearClockApproximation} or as variables in its method {\ttfamily\_process\_clock\_tag()}:
\begin{table}[h]
    \centering
    \begin{tabular}{|c|l|}
        \hline
        $S_y$ & {\ttfamily self.y\_sum} \\
        $S_{xy}$ & {\ttfamily self.xy\_sum} \\
        $\sum_{i=0}^N i$ & {\ttfamily self.i\_sum} \\
        $d_{N+1}$ & {\ttfamily step} \\
        $D_{N+1}$ & {\ttfamily front\_to\_end} \\
        \hline
    \end{tabular}
\end{table}
\end{document}